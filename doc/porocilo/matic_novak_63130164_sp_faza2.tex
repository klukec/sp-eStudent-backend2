% To je predloga za poročila o domačih nalogah pri predmetih, katerih
% nosilec je Blaž Zupan. Seveda lahko tudi dodaš kakšen nov, zanimiv
% in uporaben element, ki ga v tej predlogi (še) ni. Več o LaTeX-u izveš na
% spletu, na primer na http://tobi.oetiker.ch/lshort/lshort.pdf.
%
% To predlogo lahko spremeniš v PDF dokument s pomočjo programa
% pdflatex, ki je del standardne instalacije LaTeX programov.

\documentclass[a4paper,11pt]{article}
\usepackage{a4wide}
\usepackage{fullpage}
\usepackage[utf8x]{inputenc}
\usepackage[slovene]{babel}
\selectlanguage{slovene}
\usepackage[toc,page]{appendix}
\usepackage[pdftex]{graphicx} % za slike
\usepackage{setspace}
\usepackage{color}
\definecolor{light-gray}{gray}{0.95}
\usepackage{listings} % za vključevanje kode
\usepackage{hyperref}
\renewcommand{\baselinestretch}{1.2} % za boljšo berljivost večji razmak
\renewcommand{\appendixpagename}{Priloge}

\lstset{ % nastavitve za izpis kode, sem lahko tudi kaj dodaš/spremeniš
language=Python,
basicstyle=\footnotesize,
basicstyle=\ttfamily\footnotesize\setstretch{1},
backgroundcolor=\color{light-gray},
}

\title{Izvedba strežniškega dela aplikacije \\ Spletno programiranje}
\author{Matic Novak (63130164)}
\date{\today}

\begin{document}

\maketitle

\section{Moji podatki}

Matic Novak, 63130164, 2015/2016, spletno programiranje

\section{Navodila za namestitev spletne aplikacije}

Aplikacijo sem napisal v jeziku C\# in teče v okolju .NET MVC. Za odpiranje potrebujete Visual Studio 2013. V konfiguracijski datoteki Web.config je potrebno nastaviti connection string, ki aplikacijo poveže s podatkovno bazo. Uporabil sem MySQL podatkovno bazo, skripta za namestitev je priložena. Za preslikavo tabel v razrede sem uporabil ORM ADO.NET.

\section{Uporabniški računi}

V aplikaciji sem ustvaril 3 uporabniške račune, ki imajo skupno geslo: \texttt{Test123!}
\begin{itemize}
\item student@gmail.com
\item profesor@gmail.com
\item referat@gmail.com
\end{itemize}

\section{Predpomnenje}
Aplikacijo sem nadgradil s \texttt{cachiranjem}. Tako dosegam precej boljše čase nalaganja (predvsem statičnih) strani. Predpomnenje sem nastavil na izbranih metodah. Tiste, ki vračajo statične strani (Lokacija fakultete, O aplikaciji) se shranijo v predpomnilnik za en dan. Tabele, ki se redkeje osvežujejo, se v predpomnilnik shranijo za nekaj sekund. Metode, ki vračajo specifične rezultate za posameznega uporabnika, se shranijo v lokalni (brskalnikov) predpomnilnik. Metode, ki vračajo enake poglede za vse uporabnike pa predpomnijo na strežniku. Profile predpomnenja sem shranil v konfiguracijsko datoteko, nad posameznimi metodami pa izbrani profil pokličem z ustrezno anotacijo.

\section{Testiranje}
Preizkusil sem še tehnike avtomatskega testiranja. Za funkcionalne teste sem uporabil Selenium, kjer sem napisal nekaj testnih scenarijev in jih shranil v datoteko. Ob izvedbi vrnejo status (uspeh) testiranja. Napisal sem npr. test za prijavo v aplikacijo, test za posodobitev osebnih podatkov...\newline
Regresijske in performančne teste sem izvedel z orodjem JMeter. Nastavljal sem parameter, ki odraža število hkratnih uporabnikov sistema in opazoval odzivnost. Graf sem shranil v datoteko. Hitrost nalaganja strani sem si ogledal še v Google Chrome orodju za razvijalce, v zavihku Network. Nekaj posetkov zaslona sem priložil.

\end{document}
